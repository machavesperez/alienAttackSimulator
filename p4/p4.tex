\documentclass[]{article}

\usepackage[left=2.00cm, right=2.00cm, top=2.00cm, bottom=2.00cm]{geometry}
\usepackage[spanish,es-noshorthands]{babel}
\usepackage[utf8]{inputenc} % para tildes y ñ

%opening
\title{Práctica 4. Exploración de grafos}
\author{Miguel Angel Chaves Perez \\ % mantenga las dos barras al final de la línea y este comentario
miguelangel.chavesperez@alum.uca.es \\ % mantenga las dos barras al final de la línea y este comentario
Teléfono: 675287145 \\ % mantenga las dos barras al final de la linea y este comentario
NIF: 49071750p \\ % mantenga las dos barras al final de la línea y este comentario
}


\begin{document}

\maketitle

%\begin{abstract}
%\end{abstract}

% Ejemplo de ecuación a trozos
%
%$f(i,j)=\left\{ 
%  \begin{array}{lcr}
%      i + j & si & i < j \\ % caso 1
%      i + 7 & si & i = 1 \\ % caso 2
%      2 & si & i \geq j     % caso 3
%  \end{array}
%\right.$

\begin{enumerate}
\item Comente el funcionamiento del algoritmo y describa las estructuras necesarias para llevar a cabo su implementación.




Se implementas dos estructura de la lista List<AStarNode*> para obtener lista de los nodos abiertos y cerrados:

	. La lista cerrado se guardará nodos visitados y la lista abierto los nodos que aún no ha sido visitados
	. AstarNode* curren para saber que nodo tenemos actualmente
	. current sera expandido y mirara sus hijos para luego introducirlo en la lista abierta
	. al expandir current nos encontramos con un hijo que se encuentra en la lista de abiertos, se comprobara si es o no mejor seguir con ese nodo o repercutir 		hacía el padre
	. Si es mejor se cambiara el nodo actual current por el padre
	. Se suma un valor addicional a cada celda que representa un peso mayor o menor a la defensa principal,todo esto con la ayuda de la matriz additionalCost.







\item Incluya a continuación el código fuente relevante del algoritmo.


Se crea un vector para guardar los distintos valores de las defensas y una matriz donde se almacenaran los valores máximos que se puede obtener dado un numero de defensas y ases.





\end{enumerate}

Todo el material incluido en esta memoria y en los ficheros asociados es de mi autoría o ha sido facilitado por los profesores de la asignatura. Haciendo entrega de esta práctica confirmo que he leído la normativa de la asignatura, incluido el punto que respecta al uso de material no original.

\end{document}
