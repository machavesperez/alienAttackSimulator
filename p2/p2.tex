\documentclass[]{article}

\usepackage[left=2.00cm, right=2.00cm, top=2.00cm, bottom=2.00cm]{geometry}
\usepackage[spanish,es-noshorthands]{babel}
\usepackage[utf8]{inputenc} % para tildes y ñ
\usepackage{graphicx} % para las figuras
\usepackage{xcolor}
\usepackage{listings} % para el código fuente en c++

\lstdefinestyle{customc}{
  belowcaptionskip=1\baselineskip,
  breaklines=true,
  frame=single,
  xleftmargin=\parindent,
  language=C++,
  showstringspaces=false,
  basicstyle=\footnotesize\ttfamily,
  keywordstyle=\bfseries\color{green!40!black},
  commentstyle=\itshape\color{gray!40!gray},
  identifierstyle=\color{black},
  stringstyle=\color{orange},
}
\lstset{style=customc}


%opening
\title{Práctica 2. Programación dinámica}
\author{Miguel Angel Chaves Perez \\ % mantenga las dos barras al final de la línea y este comentario
miguelangel.chavesperez@alum.uca.es \\ % mantenga las dos barras al final de la línea y este comentario
Teléfono: 675287145 \\ % mantenga las dos barras al final de la linea y este comentario
NIF: 49071750p \\ % mantenga las dos barras al final de la línea y este comentario
}


\begin{document}

\maketitle

%\begin{abstract}
%\end{abstract}

% Ejemplo de ecuación a trozos
%
%$f(i,j)=\left\{ 
%  \begin{array}{lcr}
%      i + j & si & i < j \\ % caso 1
%      i + 7 & si & i = 1 \\ % caso 2
%      2 & si & i \geq j     % caso 3
%  \end{array}
%\right.$

\begin{enumerate}
\item Formalice a continuación y describa la función que asigna un determinado valor a cada uno de los tipos de defensas.




Se implementas dos estructura de la lista List<AStarNode*> para obtener lista de los nodos abiertos y cerrados:

	. La lista cerrado se guardará nodos visitados y la lista abierto los nodos que aún no ha sido visitados
	. AstarNode* curren para saber que nodo tenemos actualmente
	. current sera expandido y mirara sus hijos para luego introducirlo en la lista abierta
	. al expandir current nos encontramos con un hijo que se encuentra en la lista de abiertos, se comprobara si es o no mejor seguir con ese nodo o repercutir 		hacía el padre
	. Si es mejor se cambiara el nodo actual current por el padre
	. Se suma un valor addicional a cada celda que representa un peso mayor o menor a la defensa principal,todo esto con la ayuda de la matriz additionalCost.







\item Describa la estructura o estructuras necesarias para representar la tabla de subproblemas resueltos.


Se crea un vector para guardar los distintos valores de las defensas y una matriz donde se almacenaran los valores máximos que se puede obtener dado un numero de defensas y ases.




\item En base a los dos ejercicios anteriores, diseñe un algoritmo que determine el máximo beneficio posible a obtener dada una combinación de defensas y \emph{ases} disponibles. Muestre a continuación el código relevante.

\begin{lstlisting}
void DEF_LIB_EXPORTED placeDefenses(bool** freeCells, int nCellsWidth, int nCellsHeight, float mapWidth, float mapHeight
              , std::list<Object*> obstacles, std::list<Defense*> defenses) 
{

  float cellWidth = mapWidth / nCellsWidth; //Calculamos el ancho de una celda
  float cellHeight = mapHeight / nCellsHeight; //Calculamos el alto de una celda
  float** valorCelda = new float*[nCellsHeight]; // Puntuación para las celdas

  //Inicializamos todas las celdas del mapa con el valor nulo
  for(int i=0; i<nCellsHeight; i++)
  {
    valorCelda[i] = new float[nCellsHeight];
    for(int j=0; j<nCellsHeight; j++)
    valorCelda[i][j] = 0;
  }

  List<Defense*>::iterator currentDefense = defenses.begin();

  //Valores de las celda para la colocacion del centro extrator
  valorCeldaExtratora(valorCelda,cellHeight,obstacles,nCellsWidth);
    
  bool asignada = false;

  //Algorizmo voraz para la colocacion de la celda extractora
  while(!asignada)//mientras que no se haya asignado el valor a la celda extratora
  {
    int coorX,coorY;

    seleccion(valorCelda, &coorX, &coorY,nCellsWidth);//Seleccionamos candidatos
    
    //puntos medios
    float puntoMedioY = coorY*cellHeight + cellHeight/2;
    float puntoMedioX = coorX*cellWidth + cellWidth/2;

    //Comprobacion si es posible esa solucion
    if(factibilidad(puntoMedioX,puntoMedioY,obstacles,defenses,(*currentDefense)->radio,mapWidth))
    {   
      (*currentDefense)->position.x = puntoMedioX;
      (*currentDefense)->position.y = puntoMedioY;
      (*currentDefense)->position.z = 0; 
      asignada = true;
    }
  }
}
\end{lstlisting}


\item Diseñe un algoritmo que recupere la combinación óptima de defensas a partir del contenido de la tabla de subproblemas resueltos. Muestre a continuación el código relevante.

Escriba aquí su respuesta al ejercicio 4.

Conjunto de candidatos: Cada celdas del mapa
Conjunto de candidatos seleccionados: Las celdas con la mayor puntación para la colocación de las defensas
Conjunto de solución: Se comprueba que las defensas y la extractora han sido colocadas y no hay mas candidatos, aunque podría ser no óptima
conjunto de selección: Se comprueba la primera celda que encuentra de menor valor en el terreno, pudiendo o no, ser la solución óptima.
Función de factibilidad: Se comprueba que la defensa o extractor se puede colocar en la celda candidata, si no, se sigue buscando mas candidatos
Función objetivo: Maximizar el tiempo transcurridos sin que los ucos invadan la defensa extractora



\end{enumerate}

Todo el material incluido en esta memoria y en los ficheros asociados es de mi autoría o ha sido facilitado por los profesores de la asignatura. Haciendo entrega de este documento confirmo que he leído la normativa de la asignatura, incluido el punto que respecta al uso de material no original.

\end{document}
